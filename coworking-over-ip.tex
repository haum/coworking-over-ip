\documentclass{beamer}

    \usepackage[utf8]{inputenc}
    \usepackage[T1]{fontenc}
    \usepackage[french]{babel}
    \usepackage{url}

    \usetheme{Singapore}

    \logo{%
            \includegraphics[scale=0.06]{logo}
    }

    \title[Coworking-oIP]{Coworking Over IP}
    \author{Outils et astuces \textit{over IP} pour le travail collaboratif}
    \institute{Mathieu (matael) Gaborit | HAUM | Ruche Numérique}
    \date{2013}

\begin{document}

\begin{frame}
\titlepage
\end{frame}

\begin{frame}
\frametitle{Au Menu}
\tableofcontents
\end{frame}

\section{Avant de commencer} % {{{1

\begin{frame}
    \frametitle{HAUM}

    \begin{itemize}
        \item HAckerspace de l'Université du Maine
        \item association de passionné(e)s de \textit{hacking} et détournements
        \item par extension, association de technophiles :)
        \item Twitter : \texttt{@haum72}
        \item Mailing-list : \url{http://lists.matael.org/mailman/listinfo/haum_hackerspace}
        \item IRC : \texttt{\#haum @ irc.freenode.org}
    \end{itemize}
\end{frame}

\begin{frame}
    \frametitle{La Ruche Numérique}

    \begin{itemize}
        \item TODO
    \end{itemize}
\end{frame}

\begin{frame}
    \frametitle{Propriétaire ou libre ?} 
\end{frame}

\section{Constat}

\begin{frame}
    \frametitle{Constat}

    \begin{itemize}[<+->]
        \item augmentation du nombre de télétravailleurs
        \item important besoin de communication avec le reste du monde
        \item insuffisance des outils \textit{standard}
        \item besoin de travailler en même temps sur un même produit...
    \end{itemize}
\end{frame}

\section{Outils standard} % {{{1

\subsection{Old School} % {{{2

\begin{frame}
    \frametitle{Depuis des outils très \textit{old school}...}

    \pause{}

    \begin{itemize}
        \item téléphone
        \item lettre (oui oui...)
        \item rencontre \textit{IRL}
    \end{itemize}
\end{frame}

\begin{frame}
    \begin{block}{Avantages}
    \begin{itemize}
        \item bien connus et maitrisés
        \item \textit{bullet proof}
        \item canaux priviligiés désservant (presque) tout le monde
    \end{itemize}
    \end{block}

    \pause{}

    \begin{block}{Inconvénients}
    \begin{itemize}
        \item pas toujours adaptés à la réalité actuelle
            \note[item]{se déplacer ou envoyer une lettre pour envoyer un document plutot qu'un mail => :S}
        \item ne permettent pas (ou peu) le travail collaboratif
        \item peuvent engendrer des frais (recommandés, etc...)
            \note[item]{à ceux qui trouvent le mail pas fiable, nous allons y revenir}
    \end{itemize}
    \end{block}

\end{frame}

\subsection{Online} % {{{2

\begin{frame}
    \frametitle{... à du standard \textit{online}....} 

    \begin{itemize}
        \item email
        \item Skype
        \item Twitter/Facebook (malheureusement)
    \end{itemize}

    \pause{}

    \begin{block}{Bilan des forces}
        \begin{itemize}
            \item bien connus aussi mais souvent mal maîtrisés
            \item permettent l'envoi à plusieurs
            \item robustes et plus ou moins éprouvés
        \end{itemize}
    \end{block}
\end{frame}

\subsection{Le reste...} % {{{2

\begin{frame}
    \frametitle{... voire de l'inconnu}

    \begin{itemize}
        \item AdobeConnect (salle de réunion over IP), propriétaire
        \item Mumble : alternative libre de chat vocal par salon
            \note[item]{Utilisé par le PP et d'autres groupes notament}
    \end{itemize}

    \pause{}

    \begin{block}{Avis}
        \begin{itemize}
            \item méconnus mais puissants
            \item flexibles et peu couteux
            \item ....
        \end{itemize}

        \pause{}

        Nous allons parler principalement de ce type d'outils
    \end{block}
\end{frame}

\subsection{Des mails...} % {{{2

\begin{frame}
    \frametitle{Mail et sécurité}

    \begin{itemize}
        \item Créé par la RFC822 en 1982
        \item joue le rôle électronique de la poste physique ... avec tous ses travers.
    \end{itemize}

    \pause{}
    
    La poste est un \textit{intérmédiaire de confiance}, le mail aussi, mais on ne le connait pas...

    \pause{}

    Il est important de :

    \begin{itemize}
        \item Signer ses message (via PGP/GPG par exemple)
        \item Chiffrer les données sensibles
        \item Rester loin des CGU obscures et dangereuses de certains fournisseurs (Google,...)
    \end{itemize}
\end{frame}

\section{Des outils moins courants} % {{{1

\begin{frame}
    \frametitle{Pourquoi toutes ces solutions ?}

    \begin{itemize}
        \item Répondre à des besoins précis
        \item
    \end{itemize}
\end{frame}

% }}}
\end{document}

